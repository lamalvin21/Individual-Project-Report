Increasing internet use has led to vast amounts of internet resources on the World Wide Web, but computational representation of this data is currently limited. This project aims to apply AI techniques to represent musical heritage data on the World Wide Web, where computational representation is scarce. In particular, this project will use the tool SPARQL Anything to generate a knowledge graph, following an ontology. The knowledge graph will represent data from an organ dataset and expanded upon by incorporating external resources. Moreover, this report details the process of knowledge graph generation, which can be applied in other contexts. Upon evaluation, the produced knowledge graph was found to be high quality, easily interpretable and scalable, but further improvements could be made. Extensions of this project could include use of external datasets and data sources to further expand the knowledge graph. 

% briefly mention the potential impact or applications of the generated knowledge graph?