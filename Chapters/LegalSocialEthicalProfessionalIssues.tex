\chapter{Legal, Social, Ethical \& Professional Issues} 
In this chapter, legal, social, ethical and professional concerns surrounding the project will be discussed. The project adhered to both the Code of Conduct issued by the British Computer Society \cite{bcs} and the FAIR Principles \cite{fairprinciples} outlined in an article discussing the reuse of scholarly data.
                                                      
\section{Code of Conduct}
\hspace{0.5cm} The project acts in public interest as it enhances knowledge discovery through the creation of a knowledge graph, providing a broader understanding of organs. All external resources used are open-source and allow for the use of data. In particular, Wikidata's license \cite{wikidatalicense} states that it is dedicated to the public domain and has no copyright. MusicBrainz license \cite{musicbrainzlicense} allows for adaption provided that credit is given. Consent has been permitted from both external sources and licenses for both websites were strictly followed. SPARQL Anything's license \cite{apachelicense} was also adhered to as the tool was used for this project, which is permitted in its Apache 2.0 license. The project aims to provide anyone from any culture or background with a means of viewing musical heritage data on the WWW without discrimination. The Polifonia project, itself, involves people from many different fields so it enables multi-disciplinary collaboration. 

The project adheres to professional competence and integrity due to knowledge being gained from various sources of information such as those stated in the \textit{Literature Review} chapter. Throughout the report, documentation and reasoning for actions are explained with appropriate citations where necessary to ensure academic integrity. Feedback throughout the project from the supervisor Dr Albert Mero{\~n}o-Pe{\~n}uela was kindly accepted whether it be constructive criticism or project advice, enabling consideration of alternate viewpoints and seeking valuable support for the work.

Relevant authorities understand the project being undertaken as part of the Polifonia project. Communication with the supervisor has allowed for a smooth process and guaranteed complete transparency for the project's duration. Definition of intellectual data such as the dataset was made explicit throughout the project to ensure that data ownership was clearly defined.

Duty to the profession was also considered throughout the project. \textit{Requirement}, \textit{Specification} and \textit{Design} chapters describing development of the project used similar software development techniques. A professional relationship was kept with all those who assisted during the project's evolution to ensure respect and integrity. Discussion of alternative tools demonstrated a commitment to continuous professional development by considering all available technologies and selecting the most appropriate option.

However, it is possible to create a malicious knowledge graph by following the same process used to generate the knowledge graph in this project. As a result, it is important to limit the distribution of this report to only trusted parties. Malevolent actors may also attempt to use this process to create knowledge graphs that disseminate false or fabricated information. Data in the knowledge graph being in Dutch may make some users feel excluded or may introduce bias from a Dutch perspective. However, the knowledge graph can be translated into the appropriate language if necessary as the knowledge graph, itself, is publicly available. Overall, the dataset revolves around organs so the produced knowledge graph should not contain anything that may cause offence.

\section{FAIR Principles}
\hspace{0.5cm} The acronym FAIR stands for: \textbf{F}indable, \textbf{A}ccessible, \textbf{I}nteroperable and \textbf{R}eusable. The article \cite{fairprinciples} was written by a diverse set of publishers to promote the sharing and reuse of research data. 

\subsection{Findable}
\hspace{0.5cm} This ensures resources used are easy to find. The project's provided resources are all publicly searchable as part of the Polifonia project, which includes the dataset and ontology. The knowledge graph generation query, itself, is in a public GitHub repository and can be found by anyone. 

\subsection{Accessible}
\hspace{0.5cm} This ensures resources are available and can be accessed with ease. The project is stored in a public GitHub repository with no restraints as are the provided ontology and dataset. A lay user of any culture or background can use the resources as they wish, given they have internet access, and is not discriminatory. A user guide in the Appendix has also been provided to help any user execute the project. 

\subsection{Interoperable}
\hspace{0.5cm} This ensures resources are flexible and can integrate with others. Alternative external resources have been incorporated into the project so further expansion is entirely achievable. Standard formats such as RDF are also used to facilitate integration with other resources. The tool SPARQL Anything was used as it is publicly available so others may also use it. 

\subsection{Reusable}
\hspace{0.5cm} This ensures resources can be used in different contexts. Although the created ontology has been refined for the project's goals, it may be reused along with the query. This report also provides ample detail for project replication if anyone wishes to do so and involves a reference list with relevant sources for further details. 
