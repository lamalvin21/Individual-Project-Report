\chapter{Legal, Social Ethical and Professional Issues} 
In this chapter, legal, social, ethical and professional concerns surrounding the project will be discussed. Throughout the project, we adhered to both the Code of Conduct issued by the British Computer Society \cite{bcs} and the FAIR Principles \cite{fairprinciples} outlined in an article discussing the reuse of scholarly data.

\section{Code of Conduct}
\hspace{0.5cm} The project acts in public interest as it enhances knowledge discovery through the creation of a knowledge graph, providing a broader understanding of organs. All external resources used are open-source and allow for the use of data. In particular, Wikidata's license states that it is dedicated to the pubic domain and there is no copyright. MusicBrainz license allows for adaption provided that credit is given. Consent from both external sources of data are allowed and credit has been attributed to the two sources. The project aims to provide anyone from any background a means of viewing musical heritage data on the WWW. The Polifonia project, itself, involves people from many different fields so it enables multi-disciplinary collaboration.  

The project adheres to professional competence and integrity due to the knowledge gained from various sources of information such as those stated in the \textit{Literature Review} section of the report. Throughout the report, documentation and reasoning for actions is explained. Feedback throughout the project from the supervisor Dr Albert Mero{\~n}o-Pe{\~n}uela was kindly accepted whether it be constructive criticism or project advice. Enabling consideration of alternate viewpoints and seek valuable support for the work. 

Relevant authority understand the project being undertaken as part of the Polifonia project. Communication with the supervisor has allowed for a smooth process and for full transparency for the project's duration. Defining of intellectual data was made explicit throughout the project to ensure that data ownership is clearly defined.

Duty to the profession was conformed to throughout the project. \textit{Requirement}, \textit{Specification} and \textit{Design} sections dedicated to development of the project was similar to methods used in software development. A professional relationship was kept with all those who assisted the projects evolution to ensure respect and integrity. Mention of other tools for this project demonstrated the continuous professional development by acknowledging all available technologies at my disposal and carefully selecting the best option. 

\section{FAIR Principles}


% add links to literature review