\chapter{Requirements}
In this chapter, I will detail the requirements for this project. 

\section{Preliminaries}
\hspace{0.5cm} These detailed the tools required for commencing implementation.
\begin{enumerate}
\item Installation of SPARQL Anything from \cite{sparqlanythinggithub} by downloading .jar file.
\item Installation of Java Virtual Machine on the computer. 
\item Access to a Command Line Interface.
\item A Integrated Development Environment.
\end{enumerate}

\section{Software Requirements}
\hspace{0.5cm} These requirements are centered towards what needs to be implemented and produced upon completion. Assessing and validating the solution is also necessary to ensure correctness.
\begin{enumerate}
\item Ontology must be correctly configured and input into SPARQL-Anything CONSTRUCT operator.
\item Knowledge graph created must correctly reflect the organ dataset and follow ontology structure.
\item Dataset must be refined to ensure correct data is being input into the knowledge graph.
\item Knowledge graph must expand on the current dataset using external links.
\item Knowledge graph must be evaluated to prove correctness and validity.
\item Speed of knowledge graph generation must be assessed.
\end{enumerate}

\section{User Requirements}
\hspace{0.5cm} These requirements specify what an external user should be able to do when using my solution.
\begin{enumerate}
\item User must be able to execute written query.
\item User must be able to select an organ to view information on. 
\item User must be able to see the knowledge graph after the query is executed.
\item User must be able to see relevant data in the knowledge graph.
\item User must be able to view other relevant external data in the knowledge graph.
\end{enumerate}

