% Past tense
\chapter{Conclusion}
In this final chapter, the project will be concluded with directions for future work or possible extensions. Limitations of the project will also be discussed. 

\section{Concluding Remarks}
\hspace{0.5cm} In general, knowledge graph generation using SPARQL Anything was successful. Generating a knowledge graph using SPARQL Anything to display data on the web was a quick and simple solution to address the lack of computationally represented data. The process for generating knowledge graphs has been an extensive procedure, but compared to other tools, was an effective and uncomplicated method for converting any dataset into a knowledge graph. 

Nonetheless, using SPARQL Anything as a tool for knowledge graph generation was challenging. Having no prior experience with SPARQL or using the tool SPARQL Anything, was difficult, so careful reading of documentation \cite{sparqlanythinggithub} and studying SPARQL 1.1, itself, using \cite{sparlbook} was required. Limited documentation surrounding SPARQL Anything was also problematic as resources surrounding the tool were scarce. Nevertheless, SPARQL Anything's Facade-X approach compensates for this problem by allowing users to use SPARQL 1.1 syntax, where resources were plentiful. SPARQL Anything's use of existing technologies such as JSON path was also helpful as there was sufficient information surrounding it. Compared to other knowledge graph generation tools such as RML \cite{rml}, SPARQL-Generate \cite{sparqlgenerate} and ChatGPT \cite{chatgptwebsite}, it provided a simple and accurate approach. 

Extensive background and context research were also essential to understanding the project scope and provided resources. This research facilitated the implementation and evaluation phases when generating the knowledge graph.

The query created to generate the knowledge graph was evaluated using both qualitative and quantitative methods to ensure its quality and measure scalability. Critiquing the solution was vital to provide an unbiased review, so improvements could be made in the future. Evaluation revealed that knowledge graph quality was adequate but improvements, as always, could still be made and quantitative tests showed scalability to a certain extent. 

If additional time was permitted during the course of this project, more sections of the provided ontology would be used- possibly with external organ datasets or other valid resources. 

\section{Limitations}
\hspace{0.5cm} Limitations of the project have been considered for the duration of this project. Being aware of the project's limitations is vital to ensure readers intending to extend this work are knowledgeable of its bounds. These limitations are split into subsections and explicitly listed below:

\subsection{General Limitations}
\begin{itemize}
    \item \textbf{Software reliability on external factors.} Maintenance factors identified in \textit{Implementation 8.6} section may limit the query's ability to execute in the future. Deprecation or significant changes in tools used may render the query ineffective and extension of the project may no longer be feasible. 
    \item \textbf{Bias being introduced from the provided resources.} The provided dataset and ontology may include pre-existing biases as they may only include data that has been specially curated for the Polifonia project. 
    \item \textbf{Wikipedia pages not existing for all external custom links.} Due to the particularity of some data in the dataset, custom links created may not always produce populated Wikipedia pages. However, relevant recommended Wikipedia pages can be seen if it has not been created yet.  
\end{itemize}

\subsection{Evaluation Identified Limitations}
\begin{itemize}
    \item To ensure accuracy of the knowledge graph produced, it will be necessary to maintain and update the dataset on a regular basis.
    \item Dataset is written in Dutch so is limited to external links surrounding Dutch organs. Therefore, the produced knowledge graph mostly contains Dutch data and is limited to Dutch readers. 
    \item The presence of some empty strings in the dataset indicates missing data, which may limit the knowledge graph to some extent based on the completeness of the provided dataset.
    \item  The evaluation, based on file size and number of service calls, is limited in scope and may not reflect relationships between these factors and time for larger datasets. As a result, this may impact scalability of the query used for generating the knowledge graph.
\end{itemize}

\section{Future Work \& Extensions}
\hspace{0.5cm} This section discusses future work and possible extensions of the project, which may also provide solutions to the limitations identified above. Anyone willing to continue or extend work completed in this project is welcome to take the directions detailed below, but should also be wary of the limitations noted above. 

\begin{itemize}
    \item \textbf{Extension:} \\ Use of external datasets from websites such as Kaggle \cite{kaggle}, Google Dataset Search \cite{googledatasetsearch} and many others, could further extend the knowledge graph and potentially provide a more detailed solution. Finding external datasets, however, may prove challenging due to the specificity of the current dataset. 
    \item \textbf{Extension:} \\ Use of DBpedia \cite{organdbpedia} as an external data source, similar to how Wikidata and MusicBrainz were used. However, when researching its potential use, external links were not as relevant as the selected options. 
    \item \textbf{Extensions:} \\ Use the provided ontology to its fullest capability by creating external data sources. Creation of relevant data files by extracting information from the WWW may also supplement the existing dataset and will leverage the full potential of the provided ontology. 
    \item \textbf{Future Work:} \\ Application of Natural Language Processing techniques can be used to extract and integrate relevant information from the WWW into the knowledge graph. Gaps in the dataset can also be addressed by identifying relevant information to that organ on the WWW. This may also present the opportunity to explore datasets in languages other than Dutch, which can broaden the scope of datasets that can be used. A recent article \cite{DESSI2021253} details this process of generating knowledge graphs using Natural Language Processing, which may prove useful. 
    \item \textbf{Future Work:} \\ Develop a user interface or visualisation tool that enables user interaction with the generated knowledge graph by providing a visual display. This will aid understandability of the knowledge graph as well as make it more intuitive from a user's perspective. It will also help lay users who want to use or interpret the knowledge graphs for themselves. An example of data visualisation can be seen in an existing tool `QueDI' that allows lay users to easily query knowledge graphs. More detail on this tool can be found in an article \cite{de2020quedi}. Alternatively, knowledge graph visualisation frameworks already exist and may be used to aid data visualisation, for example, \cite{9355442}.
    \item \textbf{Future Work:} \\ Evaluate the knowledge graph using much larger values than those used to measure scalability in the \textit{Evaluation} section for both extremely large datasets and large amounts of service calls. Alternate methods of evaluation can also be explored such as the measurement of other metrics.
\end{itemize}

Those looking to extend or replicate this project should pay close attention to SPARQL Anything \cite{sparqlanythinggithub} documentation and gain a comprehensive understanding of all available resources. Familiarity with the dataset and a broad understanding of the semantic web, similar to this project's \textit{Context} and \textit{Background} chapters, would also be advised. Adequate planning such as ontology definition and data preprocessing may be necessary depending on the project. Software maintenance issues in \textit{Implementation 8.6} section must also be considered for long-term use. They should also be aware of the knowledge graph evaluation metrics discussed in \textit{Evaluation 9.3} section and continue to adhere to the Code of Conduct \cite{bcs} and FAIR Principles \cite{fairprinciples} throughout development. 
