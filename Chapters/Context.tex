\chapter{Context}
In this chapter, I will detail the context of my dataset and project. I have provided an explanation of all the relevant information included in my dataset or classes included in the organ dataset. I also gave a brief introduction to the tool I will be using to complete this project: SPARQL-Anything. 

\section{Organs}
\hspace{0.5cm} The dataset that has been provided revolves around the musical instrument: Organs. Mozart once described organs as:

\begin{quote}
    \textit{"The king of instruments."}
\end{quote}

An organ can come in various different sizes ranging from the size of a small upright piano to a large structure consisting of many sub-systems. In the following chapter, I will provide a description of this instrument.

\subsection{A Brief History}
\hspace{0.5cm} The first known creation of an organ was discovered in 300BC by the Greeks. These organs were run mainly using water mechanisms to produce sound because water would force air into the pipes of the organ. A particular organist at the time would play using pedals rather than a keyboard due to the simplicity of the structure. Given the time period, however, these early organs were relatively complex and were often used to demonstrate hydraulic principles as well. \cite{organhistory}

During the 14th century, organs had evolved from just the use of pedals and saw the development of a keyboard so a larger range of sounds could be made. This was accommodated with the creation of new pipes so having access to play more of these unique sounds was beneficial. These organs laid the building blocks for the organs we recognise today. \cite{organmedivalhistory}

Due to the modernisation of the world, creation of organs has become a lot easier allowing them to become more complex. Recent organ builders have had access to a plethora of materials and tools required to create these large structures with complex systems. Being able to study and explore different types of organs with ease, also aided the development of such structures.  \cite{organhistory}

\noindent \subsection{How Organs Work:}
\hspace{0.5cm} Operation of the organ involves generating and directing airflor to the requested pipe this, in turn, sound is produced when these pipes vibrate. The organist uses the keyboard and pedals to control the flow of air to pipes and draws stops to combine different sets of pipes to produce unique, blended sounds.  \cite{organvideo}

\noindent Generally, the organ system works as follows:
\begin{enumerate}
\item Note is played on keyboard.
\item Pressurised air is passed through organ system.
\item When a particular stop is pulled, an internal slider is moved allowing air to pass through pipes.
\item Sound for that particular key and stop is produced.
\end{enumerate}

\noindent \textbf{Pipe Types:}
\begin{itemize}
\item \underline{Reed Pipes}: Passes air through a reed (similar to a clarinet).
\item \underline{Flue Pipes}: Forces air through a pipe (similar to recorder).
\end{itemize}
\cite{organvideo}

\noindent \textbf{Stops:}
\\ \hspace*{0.5cm}Drawing/selecting a stop gives the player access to that stop's set of pipes available (for a particular key on the keyboard). 
\begin{itemize}
\item Draw Stops
\item Tabs
\end{itemize}

Multiple stops can be drawn at once in the same division and multiple pipes will produce sound (pressurised air will be passed through them) from the press of one key. \cite{organvideo}

\medskip
\noindent \textbf{Rank:}
\\ \hspace*{0.5cm} A rank is a row of pipes (controlled by a single stop) that is part of the organ which makes the same musical sound but does so at different pitches. 

For example, a rank of spire flue pipes all produce the same wind instrument sound, but each key pressed will produce a different flue pitch.  \cite{organvideo}

\medskip
\noindent \textbf{Divisions:}
\\ \hspace*{0.5cm} An organ division is a set of pipes found within an organ that are controlled by a keyboard or the pedals. Stops of an organ are arranged into divisions, which are usually given unique names. For example, some divisions include:
\begin{itemize}
\item Swell (smaller pipes).
\item Great (larger pipes).
\item Pedal.
\item Choir.
\item Positiv
\end{itemize}

All of these divisions usually accompany whatever is deemed appropriate for the given context. For example, if you are accompanying a choir, using the choir division would be best so as to not drown out the voices of the choir.  \cite{organvideo}

\medskip
\noindent \textbf{Coupler:}
\\ \hspace*{0.5cm} A coupler allows the stops of one division to be played from the keyboard of another division- the combining of divisions. For example, the "Great" division is usually played on the manual (keyboard) of the organ, but pulling the "Great to Pedal" coupler can allow for this division to be played on the pedals.

It is worth noting that couplers may look like stops and need to be pulled to be activated, but are not stops. \cite{organvideo}

\medskip
\noindent \textbf{Wind System:}
\\ \hspace*{0.5cm} Wind systems are responsible for:
\begin{itemize}
\item Producing
\item Storing
\item Delivering
\end{itemize}
air within the organ system.

The organ wind system is a subsystem of the organ, which is responsible for delivering air to the organ's different pipes. Wind is generated within the system and then directed through a series of channels to the pipes of the organ. Upon delivery of air to each individual pipe, a sound is made based on the key or pedal pressed. \cite{organvideo}

\subsection{Organ Ontology}
\hspace{0.5cm} The organ ontology provided the framework of the knowledge graph I am going to create using the organ dataset. This covers the dataset as well as other details about an organ so that all the data can be displayed in the form of a knowledge graph, following the structure of the ontology. 

The ontology includes classes for the organ system:
\begin{itemize}
\item Organ Console
\item Organ Wind System
\item Organ Case
\item Organ Action
\item Organ Division
\end{itemize}

But the ontology also includes classes about a particular organ's owner and offers background on a given organ. For example:
\begin{itemize}
\item Description
\item Agent and the role of this agent (owner, organist, builder etc.)
\end{itemize}

\section{Polifonia Project}
\hspace{0.5cm} The Polifonia project is a European project funded by the EU Horizon 2020 Programme that will manifest the connections between musical heritage from the 16th century to the present day \cite{polifonia}. The project involves many different types of experts from musicologist to computer scientists to attempt to find links between different parts of musical cultural heritage. 

The main aim of this project is to combine computational tools and methods to access and manipulate musical cultural heritage on the WWW. Another aim is to use these computational methods in a different context (outside of Computer Science field) and test whether such techniques can be adequately applied elsewhere. Hence, the involvement of experts from all over Europe to guide the progress of this project. \cite{polifoniaproject}

The section my project is involved in is the creation of a knowledge graph in the context of pipe organs. The dataset was provided and previously put into an organised format by a musicologist as part of this project. The general organ ontology was also created prior to my project commence but requires refinement and some changes in order to create the knowledge graph. Due to the project being European-based, the data stored in the dataset is in Dutch and the knowledge graph created will also produce dutch text or terms.  

\subsection{Organ Dataset}
\hspace{0.5cm} The organ dataset provided describes the various specifics about a given organ. The dataset itself is split into many different folders and it groups specific details about an organ in one .JSON file. Each .JSON file contains information about many different organs, which can be uniquely identified by their unique id. The structure of the .JSON files follow the same format: an organs and the corresponding details about each organ in the form of a .JSON. An example of two files "base.json" and "history\_base.json" can be seen below. 

\begin{lstlisting}
    # base.json extract
    {
      "Part01_001MIDDE": {                             # Unique Organ Code
        "building": "Koorkerk",                        # Basic organ details
        "monumentnumber": "28671",
        "name": "Middelburg, Koorkerk",
        "organnumber": "971",
        "place": "Middelburg",
        "whichorgan": "",
        "year": "1479"
      },
      "Part01_002UTREJ": {
        "building": "Jacobikerk",
        "monumentnumber": "36148",
        "name": "Utrecht, Jacobikerk",
        "organnumber": "1514",
        "place": "Utrecht",
        "whichorgan": "",
        "year": "1509"
      },
      ... # Other organs
    }

    #history_base.json extract
    {
      "Part01_001MIDDE": {                              # Unique Organ Code
        "builder": "Peter Gerritsz",                    # Basic history of organ
        "originallocation": "Utrecht, Nicolaikerk",
        "year": "1479"
      },
      "Part01_002UTREJ": {
        "builder": "Gerrit Petersz",
        "originallocation": "",
        "year": "1509"
      },
      ... # Other organs
    }
\end{lstlisting}

The data is split into two different files and the unique organ name, with it's grouped relevant data, appear.  

\section{SPARQL Anything}
\hspace{0.5cm} SPARQL Anything acts as an extension of SPARQL and allows knowledge graphs to be generated from a dataset. SPARQL Anything uses Facade-X, which is similar to the Facade design pattern. The facade design pattern is a useful way to improve the usability of a complex system by providing a simpler and more intuitive interface to it. In this case, SPARQL Anything masks the complex components by allowing the user to continue to use SPARQL syntax. Using SPARQL Anything, the SERVICE operator can  be overridden and we can use SPARQL Anything within a query. \cite{sparqlanything}

The CONSTRUCT operator allows knowledge graphs to be generated, through the framework of a given ontology. \cite{sparqlanythinggithub}

SPARQL Anything allows us to query a dataset to suit our needs and generate a knowledge graph using an ontology framework all in one tool. \cite{sparqlanythinggithub}

\section{Other Tools}

\hspace{0.5cm} In this section, I will detail another way of generating knowledge graphs in the context of organs, describing  the approach as well as the positives and negatives of suggested approach.

\subsubsection{ChatGPT}
\hspace{0.5cm} ChatGPT is a recently developed artificially intelligent chatbot that answers questions in a conversational way. The chatbot specialises in being casual, so it is able to answer appropriate follow-up questions and respond to complex questions in a way that is easy to understand. ChatGPT uses AI techniques such as reinforcement learning and supervised learning to respond correctly to questions. \cite{chatgptwiki} This chatbot released it's prototype in November 2022 and is currently free to use. Uses of ChatGPT include:
\begin{itemize}
\item Solving progamming questions.
\item Fixing bugs in code.
\end{itemize}

When ChatGPT is asked to create a knowledge graph regarding organs, this is what it produces:

\lstset
{
    breaklines=true,
    breakatwhitespace=true,
}
\begin{lstlisting}
    Nodes: Keyboard, Pedals, Pipes, Blower, Stops
    Edges:
        -  The keyboard is connected to the pipes, indicating that pressing keys on the keyboard causes the corresponding pipes to produce sound.
        -  The pedals are connected to the pipes, indicating that pressing the pedals can also cause certain pipes to produce sound.
        -  The blower is connected to the pipes, indicating that the blower supplies air to the pipes, allowing them to produce sound.
        -  The stops are connected to the pipes, indicating that the stops control which pipes are active and can produce sound.
\end{lstlisting}

This response represents a small knowledge graph mainly comprised of organ parts and how it produces sound. The output of the knowledge graph is in plain text so has to be manually read and interpreted, so is not in machine-readable format. 

This approach is a very straightforward technique for generating an organ-related knowledge graph and makes use of the strengths of ChatGPT. The knowledge graph, although in prose, is very useful and easy to understand due to ChatGPT's ability to communicate conversationally. The information provided is based on the chatbot's general understanding of the organ and is consistent with the instrument's details. The nodes and edges in the graph that are stated in the knowledge graph are also consistent with the general information regarding an organ. 

However, this approach generates a small knowledge graph and has to be interpreted correctly in prose. In terms of the produced knowledge graph's validity, the knowledge graph does not include any real-world data, but more general topics regarding organs so the generated knowledge graph is a lot closer to an ontology than a real-world knowledge graph. The limitations of ChatGPT include that it is restricted to it's input domain \cite{chatgptwebsite}, so real-world data can not be found and passed into this ontology to create a suitable knowledge graph. 

In conclusion, the knowledge graph produced by ChatGPT is closer to an ontology about organs. The limitations surrounding ChatGPT do confine it to providing a generic solution, but if it can evolve to find a way to gather real-world data online using the Semantic Web, for instance, the knowledge graph created would be a viable solution (given the provided ontology).