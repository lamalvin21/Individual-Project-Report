\chapter{Context}
This chapter will contextualise the dataset and project, providing reasoning for all relevant information included in the organ dataset. Also, a brief introduction to the tool used to complete this project: SPARQL-Anything as well as reasoning for using it over other tools is provided. 

\section{Organs}
\hspace{0.5cm} The dataset that has been provided revolves around the musical instrument: Organs. Mozart once described organs as:

\begin{quote}
    \textit{"The king of instruments."}
\end{quote}

An organ can come in various different sizes ranging from a small upright piano to a large structure consisting of many sub-systems. The following chapter will provide a description of the instrument's components.

\subsection{A Brief History}
\hspace{0.5cm} 
The inventor of the organ is accredited by many sources to an engineer in Alexandria during the third century BC called Ctesibius. In the first century AD, organs used water and hydraulic engineering to generate sound. \cite{organhistory}

During the 14th century, organs had evolved from the use of just pedals and saw the development of a keyboard so a larger range of sounds could be made. This was accommodated with the creation of new pipes so having access to more of these unique sounds was beneficial. These organs laid the building blocks for the organs we recognise today. \cite{organmedivalhistory}

Due to the modernisation of the world, creation of organs became a lot easier allowing them to become more complex. Recent organ builders have had access to a plethora of materials and tools required to create these large structures with complex systems. Being able to study and explore different types of organs with ease, also aided the development of such structures.  \cite{organhistory1}

\noindent \subsection{How Organs Work:}
\hspace{0.5cm} Operation of the organ involves generating and directing airflow to the requested pipe causing them to vibrate and produce sound. The organist uses the keyboard and pedals to control the flow of air to pipes and draws stops to combine different sets of pipes to produce unique, blended sounds.  \cite{organvideo}

\noindent Generally, the organ system works as follows:
\begin{enumerate}
\item Note is played on keyboard.
\item Pressurised air is passed through organ system.
\item When a particular stop is pulled, an internal slider is moved allowing air to pass through pipes.
\item Sound for that particular key and stop is produced.
\end{enumerate}

\noindent \textbf{Pipe Types:}
\begin{itemize}
\item \underline{Reed Pipes}: Passes air through a reed (similar to a clarinet).
\item \underline{Flue Pipes}: Forces air through a pipe (similar to recorder).
\end{itemize}
\cite{organvideo}

\noindent \textbf{Stops:}
\\ \hspace*{0.5cm}Drawing/selecting a stop gives the player access to that stop's set of pipes available (for a particular key on the keyboard). Types of stops include:
\begin{itemize}
\item Draw Stops
\item Tabs
\end{itemize}

Multiple stops can be drawn at once in the same division and multiple pipes will produce sound (pressurised air will be passed through them) from the press of one key. \cite{organvideo}

\medskip
\noindent \textbf{Rank:}
\\ \hspace*{0.5cm} A rank is a row of pipes (controlled by a single stop) that produces the same musical sound but does so at different pitches. 

For example, a rank of spire flue pipes all produce the same wind instrument sound, but each key pressed will produce a different flue pitch.  \cite{organvideo}

\medskip
\noindent \textbf{Divisions:}
\\ \hspace*{0.5cm} An organ division is a set of pipes found within an organ controlled by the keyboard or pedals. Stops of an organ are arranged into divisions, which are usually given unique names. For example, some divisions include:
\begin{itemize}
\item Swell (smaller pipes).
\item Great (larger pipes).
\item Pedal.
\item Choir.
\item Positiv
\end{itemize}

All of these divisions usually accompany whatever is deemed appropriate for the given context. For example, when accompanying a choir, using the choir division would be suitable so as to not drown out the voices of the choir. \cite{organvideo}

\medskip
\noindent \textbf{Coupler:}
\\ \hspace*{0.5cm} A coupler allows the stops of one division to be played from the keyboard of another division (i.e. the combining of divisions). For example, the "Great" division is usually played on the manual (keyboard) of the organ, but pulling the "Great to Pedal" coupler can allow for this division to be played on the pedals. \cite{organvideo}

\medskip
\noindent \textbf{Wind System:}
\\ \hspace*{0.5cm} Wind systems are responsible for:
\begin{itemize}
\item Producing
\item Storing
\item Delivering
\end{itemize}
air within the organ system.

The organ wind system is a subsystem of the organ, which is responsible for delivering air to the organ's different pipes. Wind is generated within the system and then directed through a series of channels to the organ's pipes. Upon delivery of air to each individual pipe, a sound is made based on the key or pedal pressed. \cite{organvideo}

\section{Polifonia Project}
\hspace{0.5cm} The Polifonia project is a European project funded by the EU Horizon 2020 Programme that will manifest the connections between musical heritage from the 16th century to the present day \cite{polifonia}. The project involves many different types of experts from musicologist to computer scientists to attempt to find links between different parts of musical cultural heritage. 

The main aim of this project is to combine computational tools and methods to access and manipulate musical cultural heritage on the WWW. Another aim is to use these computational methods in a different context (independent of Computer Science field) and test whether such techniques can be adequately applied elsewhere. Hence, the involvement of experts from all over Europe to guide the progress of this project. \cite{polifoniaproject}

This project is involved in the creation of a knowledge graph in the context of organs. The dataset was provided and previously put into an organised format by a musicologist as part of this project. The general organ ontology was also created prior to this project's commence but requires refinement and some changes in order to create the knowledge graph. Due to the project being European-based, the data stored in the dataset is in Dutch and the knowledge graph created will also produce dutch text or terms.  

\subsection{Organ Dataset}
\hspace{0.5cm} The organ dataset provided describes the various specifics about a given organ. The dataset itself is split into many different folders and it groups specific details about an organ in one .JSON file. Each .JSON file contains information about many different organs, which can be uniquely identified by their unique id. The structure of the .JSON files follow the same format: an organ and the corresponding details about each organ in the form of a .JSON. An example of two files "base.json" and "history\_base.json" can be seen below. 

\begin{lstlisting}
    # base.json extract
    {
      "Part01_001MIDDE": {                             # Unique Organ Code
        "building": "Koorkerk",                        # Basic organ details
        "monumentnumber": "28671",
        "name": "Middelburg, Koorkerk",
        "organnumber": "971",
        "place": "Middelburg",
        "whichorgan": "",
        "year": "1479"
      },
      "Part01_002UTREJ": {
        "building": "Jacobikerk",
        "monumentnumber": "36148",
        "name": "Utrecht, Jacobikerk",
        "organnumber": "1514",
        "place": "Utrecht",
        "whichorgan": "",
        "year": "1509"
      },
      ... # Other organs
    }

    #history_base.json extract
    {
      "Part01_001MIDDE": {                              # Unique Organ Code
        "builder": "Peter Gerritsz",                    # Basic history of organ
        "originallocation": "Utrecht, Nicolaikerk",
        "year": "1479"
      },
      "Part01_002UTREJ": {
        "builder": "Gerrit Petersz",
        "originallocation": "",
        "year": "1509"
      },
      ... # Other organs
    }
\end{lstlisting}

The data is split into multiple different files containing the unique organ name with it's relevant grouped data. The dataset contains data in Dutch because the Polifonia Project is a European project.

\subsection{Organ Ontology}
\hspace{0.5cm} The provided organ ontology contains all the relevant information relating to an organ, ranging from it's location to the parts within it. 

The ontology includes classes for the organ system:
\begin{itemize}
\item Organ Console
\item Organ Wind System
\item Organ Case
\item Organ Action
\item Organ Division
\end{itemize}

But the ontology also includes classes about a particular organ's owner and offers background on a given organ. For example:
\begin{itemize}
\item Description
\item Agent and the role of this agent (owner, organist, builder etc.)
\end{itemize}

The ontology is extremely vast and may contain nodes or relationships that do not appear within the dataset. Therefore, refinement and adjustment of the ontology to fit the needs of the provided dataset is required. Carefully selecting parts of the ontology to act as the framework of the resulting knowledge graph is vital to ensure all relevant data in the dataset is accurately represented. 

\section{SPARQL Anything}
\hspace{0.5cm} The tool used to generate the knowledge graph is SPARQL Anything. SPARQL Anything acts as an extension of SPARQL and allows the generation knowledge graphs from a dataset. SPARQL Anything uses Facade-X, which is similar to the Facade design pattern. The facade design pattern improves the usability of a complex system by providing a simpler and more intuitive interface to it. In this case, SPARQL Anything masks the complex components by allowing the user to continue to use SPARQL, so familiar functions such as BIND() or CONCAT() can be used allowing users to generate knowledge graphs easily. Therefore, the time required to learn how to use such a tool would be short as there is an abundance of resources available related to learning SPARQL, in contrast to other tools which may require learning a new language. 

Using SPARQL Anything, the SERVICE operator can be overridden and we can call SPARQL Anything within a query. \cite{sparqlanything}

The CONSTRUCT operator allows knowledge graphs to be generated, by using the framework of a given ontology. \cite{sparqlanythinggithub}

\section{Other Tools}

\hspace{0.5cm} This section, will detail alternate methods of generating knowledge graphs describing the approach as well as the positives and negatives of suggested approaches.

\subsubsection{ChatGPT}
\hspace{0.5cm}  ChatGPT is a machine-learning-based tool developed by OpenAI and is an interactive chatbot. Answers provided by the chatGPT are conversational so answers are easy to understand and is suggested to be the best chatbot to date. \cite{chatgpt} OpenAi released a prototype chatbot on 30th November 2022 and is currently free to use. Uses of ChatGPT include:
\begin{itemize}
\item Solving programming questions.
\item Fixing bugs in code.
\end{itemize}

When asked to create a knowledge graph regarding organs, ChatGPT produced:

\lstset
{
    breaklines=true,
    breakatwhitespace=true,
}
\begin{lstlisting}
    Nodes: Keyboard, Pedals, Pipes, Blower, Stops
    Edges:
        -  The keyboard is connected to the pipes, indicating that pressing keys on the keyboard causes the corresponding pipes to produce sound.
        -  The pedals are connected to the pipes, indicating that pressing the pedals can also cause certain pipes to produce sound.
        -  The blower is connected to the pipes, indicating that the blower supplies air to the pipes, allowing them to produce sound.
        -  The stops are connected to the pipes, indicating that the stops control which pipes are active and can produce sound.
\end{lstlisting}

This response represents a small knowledge graph mainly comprised of organ parts and how it produces sound. The output of the knowledge graph is in plain text so has to be manually read and interpreted, so is not in machine-readable format. 

This approach is a very straightforward technique for generating an organ-related knowledge graph and makes use of the strengths of ChatGPT. The knowledge graph, although in prose, is very useful and easy to interpret due to ChatGPT's ability to communicate conversationally. The information provided is based on the chatbot's general understanding of the organ and is consistent with the instrument's details. The nodes and edges stated in the knowledge graph are also consistent with the elements of an organ. 

However, this approach generates a small knowledge graph and has to be interpreted correctly in prose. In terms of the produced knowledge graph's validity, the knowledge graph does not include any real-world data, but more general topics regarding organs so the generated knowledge graph is a lot closer to an ontology than a real-world knowledge graph. The limitations of ChatGPT include that it is restricted to it's input domain \cite{chatgptwebsite}, so real-world data can not be found and passed into this ontology to create an accurate knowledge graph. 

In conclusion, the knowledge graph produced by ChatGPT is closer to an organ ontology. The limitations surrounding ChatGPT do confine it to providing a generic solution, but if it can evolve to find a way to gather real-world data online using the Semantic Web, for instance, the knowledge graph created would be a viable solution (given the provided ontology).

\subsubsection{SPARQL-Generate and RML}
\hspace{0.5cm} Other solutions similar to SPARQL Anything's approach and implementation include: SPARQL-Generate and RML. But, the reasoning for selecting SPARQL-Anything over the other options will be detailed below.

SPARQL-Generate and RML do not support Metadata or Embedded data formats. \cite{sparqlanything} SPARQL Anything, therefore, allows more flexibility during implementation and has a broader scope. 

SPARQL-Generate is an extension of SPARQL so requires the studying of the newly created extension. RML requires a mapping to RDF using a new R2RML vocabulary, so again requires studying a new language. SPARQL Anything, on the other hand, because it uses Facade-X, is reliant on SPARQL 1.1 knowledge. \cite{sparqlanything} This results in less time spent studying something completely new and has easier learnability given the vast amount of resources available around SPARQL.

In terms of complexity, SPARQL Anything performs better in experiments using the number of distinct items or variables that need to be combined with the query as a measurement of complexity. \cite{sparqlanything} This is important when considering possible expansion of the knowledge graph and when assessing the speed of query execution. 