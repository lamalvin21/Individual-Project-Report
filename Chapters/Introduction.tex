\chapter{Introduction}
There are many information sources today there are still not computationally represented or explored fully on the web. Therefore, there is a demand for a means to represent this information in the form of a knowledge graph, so the vast amounts of data on the web can be available to use by anyone. Finding an automated way to generate knowledge graphs from structured data or text found on the web is vital to addressing this demand. Current solutions to generating knowledge graphs include: SPARQL-Generate \cite{sparqlgenerate}, RML \cite{rml} and ChatGPT \cite{chatgptwebsite}. However, these tools are limited by their scope and complexity to reach the solution, so selecting SPARQL-Anything \cite{sparqlanythinggithub} for implementation was most favourable. 

People who want to generate knowledge graphs from their dataset can not do so directly as the process is complex and requires understanding of knowledge engineering. Knowledge Engineering requires experience and research on areas such as what ontologies to use or create, what vocabularies to use and what classes to use. So the process of generating knowledge graphs from a dataset is not instantaneous or trivial and requires research. Generating knowledge graphs can be done without the use of a tool and can be done by hand, which is possible for smaller knowledge graphs. But for larger knowledge graphs and datasets, generating the knowledge graph triple by triple without the assistance of a tool would take a long amount of time and is prone to human error.

The project I will undertake requires the generation of knowledge graphs using a organ orientated dataset and provided ontology, which will be implemented using SPARQL-Anything. In order to start the project, knowledge of the semantic web and it's capabilities was required so that I could understand the scope of what needed to be done. Understanding and being able to interpret RDF, ontologies and knowledge graphs would be helpful to know during the course of the project. 

The project is part of the Polifonia project, which seeks to create an ecosystem of computational tools and methodologies to spread knowledge about musical history on the internet \cite{polifoniaproject}. My designated project involved a dataset revolving around organs, so background knowledge as well as organ components are detailed in the report. 


TBC...
