% Future tense
\chapter{Introduction}
A knowledge graph is a method of representing real-world knowledge in a graphical format and is an application of knowledge engineering. The field of knowledge engineering involves designing, creating and maintaining knowledge-based systems \cite{studer_benjamins_fensel_1998}. Knowledge graphs require the creation of ontologies, which is a specification of the concept \cite{Breitman2007}. The primary aim of this project is to use the knowledge graph generation tool: SPARQL Anything and a given organ-oriented dataset to create a knowledge graph; the process in which will be detailed in this report. This first chapter will introduce the rationale, aims and scope of this project. 

\section{Rationale}
\hspace{0.5cm} There are many information sources today that are still not computationally represented or fully explored on the World Wide Web (WWW). In an article \cite{bizer2011linked}, the authors (including creator of the WWW: Tim Berners-Lee) discuss substantial amounts of structured data published on the WWW but show there is ample opportunity for further exploration of other topics. Therefore, there is demand for a means of representing information, potentially through the use of knowledge graphs, to make vast amounts of web data available for computational and human use. Finding an automated method of generating knowledge graphs from structured data or text found on the WWW is vital to addressing this demand. Current solutions for generating knowledge graphs include SPARQL-Generate \cite{sparqlgenerate}, RML \cite{rml} and Large Language Models (LLMs). LLMs are machine-learning models trained by enormous amounts of data to mimic the patterns of natural language with ChatGPT \cite{chatgptwebsite} being an example of such implementation. Nonetheless, the capability of these tools is restricted by the number of input formats they can accommodate and their level of complexity in terms of learnability \cite{sparqlanything}. Consequently, selecting SPARQL Anything \cite{sparqlanythinggithub} for implementation was most favourable. 

Generating knowledge graphs directly from datasets is not trivial as the process is complex and requires an understanding of knowledge engineering. Knowledge engineering, itself, requires experience and research in areas such as: what ontologies to create or use, what vocabularies to use and what classes to use. Thus, knowledge graph generation from a dataset is not instantaneous and requires extensive research. In the case of smaller knowledge graphs, it is possible to create knowledge graphs without the use of a tool manually. Nonetheless, physically the knowledge graph triple by triple would be time-consuming and is prone to human error, especially for larger knowledge graphs and datasets. This approach is also not sustainable. 

Motivation for organ representation, in particular, revolves around the lack of detail surrounding them. Current musical culture representation is generally insufficient \cite{polifoniaproject}, thus specific information regarding organs is necessary to enable the empowerment of such a complex structure. The organ is a fundamental part of music history and a distinguished composer- Mozart lauded the instrument as ``the ruler of instruments". Furthermore, Johann Sebastian Bach, a renowned composer widely regarded as one of the greatest composers of all time, had a preference for the organ \cite{wolff2011organs}. Enriching existing representations with vast amounts of detail about organs is a powerful method of honouring such a venerable instrument. The wealth of valuable information in this emerging field of organs will, therefore, enable more effective querying and strengthen its symbolic representation. 

\section{Aims}
\hspace{0.5cm} Integration of various data sources into a single, encapsulated knowledge graph can enable the sharing and understandability of this information by both humans and machines. Knowledge graphs may provide economic advantages by reducing the time and effort to analyse vast amounts of data. Consumers may use the knowledge graphs for their own needs or follow the process outlined in the report to create their own knowledge graphs.

To the extent of our research, creating knowledge graphs for this particular subject (organs) had not yet been fully explored in this level of detail. For instance, knowledge graphs were previously developed for music recommendation systems \cite{oramas2016sound} by other researchers. So knowledge graph generation techniques had been done previously, but the employed methodology is of a different context to our problem. Existing organ knowledge graphs on Wikidata \cite{organwikidata}, MusicBrainz \cite{organmusicbrainz} and DBpedia \cite{organdbpedia} are sufficient for basic familiarity, but do not go into the same amount of depth as the knowledge graph created in this project. This project also aims to address this particular knowledge representation gap. 

\section{Scope}
\hspace{0.5cm} The project is part of the Polifonia project, which seeks to create an ecosystem of computational tools and methodologies to spread knowledge about musical history on the internet \cite{polifoniaproject}. Therefore, this report assists the Polifonia project in achieving its objectives and facilitates its progression. This designated project involves a dataset centred around organs so the report includes a breakdown of the various organ components as well as background information regarding technical knowledge. 

This report will detail the process of knowledge graph generation using an organ-orientated dataset and ontology \cite{organontology}, both of which are contextualised to understand the dataset fully and to refine the provided ontology. In order to commence, knowledge of the semantic web and its capabilities are required so as to understand the scope of the project. Researching and being able to interpret RDF (a standard model for data interchange on the web used for representing and exchanging metadata), ontologies and knowledge graphs is also required to reach the solution as well as interpret it. 

The following chapters provide a comprehensive view of the design and implementation process for generating knowledge graphs. The \textit{Design} chapter complements the implementation phase by using visual tools and careful planning through software development techniques. Subsequently, the \textit{Implementation} chapter meticulously explains the steps required to reach the final solution. A thorough evaluation of the produced knowledge graph is performed with both qualitative and quantitative forms of assessment being carried out. The legal, social, ethical and professional issues surrounding this project are also reviewed to ensure adherence to the Code of Conduct \& Code of Good Practice issued by the British Computer Society \cite{bcs} as well as the FAIR Principles \cite{fairprinciples}. This is also intended to assist in identifying any potential issues that may arise during the project.

Finally, the conclusion summarises work completed during the course of this project and includes a discussion on future work and limitations. 
