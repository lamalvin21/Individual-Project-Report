\chapter{Specification}
This chapter will go into more detail regarding each of the requirements specified in the previous chapter. More specifically, this chapter will provide an explanation on how each requirement can be achieved.

\section{Software Requirement Specification}
\begin{longtable}{|p{2.25cm}|p{4.5cm}|p{6.5cm}|}
\hline
\textbf{Requirement No.} & \textbf{Requirement} & \textbf{Specification}\\
\hline

1& 
Ontology must be correctly configured and input into the query's CONSTRUCT clause. &
Organ ontology created will match data from the given dataset and only relevant segments of the provided ontology will be included. To ensure the generation of a valid knowledge graph, it is important to extract relevant parts of the ontology and define the correct types. This guarantees that all classes are properly converted and ensures all relationships are accounted for with no redundant details. \\
\hline

2&
The knowledge graph created must correctly reflect the organ dataset and follow ontology structure. &
Identify data from the dataset in the knowledge graph and ensure relationships are consistent. Compare ontology with produced knowledge graph and ensure the framework is being followed. Override the SERVICE keyword in the query to use services of SPARQL Anything. \\
\hline

3&
The dataset must be refined to ensure correct data is being input into the knowledge graph. &
Query correctly navigates through the dataset to find the correct data to add to the knowledge graph. This can be done through string manipulation in the query to generate the correct path, thus obtaining the right data. In the WHERE clause of the query, ensure data being input into the knowledge graph is understandable and concise (if not already).  \\
\hline

4&
The knowledge graph must expand on the current dataset using external links. &
Search for external links to expand the knowledge graph beyond the dataset and present additional relevant information. Use web resources such as WikiData, MusicBrainz and DBpedia to find potential expansion points of the knowledge graph derived from the organ dataset. Use multiple SERVICE calls (within the WHERE clause) to obtain external data and add it to the current knowledge graph. String manipulation and string-to-link conversion (IRI) may be necessary to add external links. \\
\hline

5&
Knowledge graphs must be evaluated to prove correctness and validity. &
Validate correctness of the knowledge graph produced. Look into accuracy, coverage and coherency of the produced knowledge graph and analyse the different aspects of those areas in more detail. \cite{knowledgegraphevaulationbook} and \cite{evaluationpaper}, identified in the \textit{Literature Review} Chapter, will be used to help during evaluation. \\ 
\hline

6&
Scalability of knowledge graph generation must be assessed. &
Assess speed of the query for knowledge graph generation to ensure expansion of the knowledge graph is possible in the future without encountering severe time constraints. This will measure scalability based on many different factors (more detail in the \textit{Evaluation} chapter). 
\hline
\caption{Software Requirement Specification Table}
\end{longtable}

\begin{table}[h!]
\section{User Requirement Specification}
\begin{center}
\begin{tabular}{c|p{2in}p{2.55in}}
Requirement \\ No.&Requirement&Specification\\\hline 

1&
User must be able to execute written query. & 
Upon downloading the GitHub repository and the SPARQL Anything executable, a user should be able to execute the query using SPARQL Anything and have it produce a knowledge graph. \\
\hline

2& 
User must be able to select an organ to view information on. &
When writing the command on the command line to execute the query, the user can be able to input an organ that they want to view specific information on by passing it as a parameter. The resulting knowledge graph is then specific to the requested organ. \\
\hline

3&
User must be able to see the knowledge graph following query execution. &
Upon executing the query, a knowledge graph should be visible in the form of a TTL file. The user can also see the knowledge graph on command line if they wish.\\
\hline

4&
User must be able to see relevant data in the knowledge graph. & 
The knowledge graph must display the correct data and relationships between them. When viewing the final solution, the user can compare data from the dataset and knowledge graph. Organ-specific data can be seen for the requested organ.\\
\hline

5&
User must be able to view other relevant external data in the knowledge graph. & 
The user should be able to view external links to data in the knowledge graph and acknowledge the relationship between data from the dataset and external data. External links unique to the requested organ can also be observed. 

\end{tabular}
\end{center}
\caption{User Requirement Specification Table}
\end{table}

