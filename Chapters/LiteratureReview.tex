\chapter{Literature Review}
In this chapter, resources (i.e. articles, websites, videos or books) being studied are discussed. The literature reviewed encompassed a broad range of topics to facilitate understanding of the context and ensure successful completion of the project. All references are mentioned for readers to carry out more research if they wish. 

\section{Background and context information sources}
\hspace{0.5cm} Research surrounding the semantic web and its technologies was required in order to provide a broad context for the project and its implementation. The article \cite{semanticweb} offered a gentle introduction to the topic although not directly relevant to the project scope. Limited resources specifically related to the project's context meant that resources of different contexts had to be explored. A book \cite{berners-TBLBook} co-authored by the creator of the WWW provided valuable insights as the concept of the semantic web originated from him, making it beneficial to understand his perspective on the subject. Other articles \cite{bizer2011linked} were explored to understand the current status of the semantic web and identify areas for further research. 

Targeting articles that offered an introduction to the topic were favoured to help build a basic understanding. However, they were limited by their introductory nature and lack of in-depth detail. Using articles such as \cite{ontology}, \cite{knowledgegraph} and \cite{rdf} helped understanding of relevant background information. All three articles introduced their topics sufficiently and for further research into the practical uses of knowledge graphs, \cite{searchengine} and \cite{oramas2016sound} were used. Fully grasping \cite{oramas2016sound}, in particular, was vital due to its relevance in the field of music. Websites from W3C \cite{TTL}, \cite{w3crdf} and \cite{w3cvocabularies} also facilitated research by filling gaps in understanding. Being a reputable international standard organisation and involving a wide variety of members displays the source's credibility. However, from a lay reader's perspective, reading these sites could pose a challenge due to usage of technical language. 

Research regarding organs was necessary to gain an extensive understanding of the provided dataset. Starting with the history of organs required the use of \cite{organhistory}, \cite{organhistory1}, \cite{organmedivalhistory}. This involved a book as well as some online material to help provide a short historical summary of organs and their development over time. Extraction of relevant information from each source was necessary to provide a brief historical context. Specific sources regarding the history of organs were scarce and others found \cite{apel1948early} and \cite{ochse1988history} relied heavily on previous understanding of music history so the selected sources of information were best from a lay reader's perspective. Different mediums were used particularly to understand the different parts of the organ such as videos \cite{organvideo} and \cite{organvideo1}. This aided visualisation of the components in the dataset from experts in the field. Alternate introductory sources were limited so using videos was sufficient. 

Details regarding the Polifonia project's purpose and contextual information were acquired from its specification \cite{polifoniaproject} and the official website \cite{polifonia}. These websites provided nuanced detail regarding the Polifonia project, which were the few resources available regarding the project. Bias may be introduced as all resources are written by participants of this project. 

\section{Knowledge Graph Generation Techniques}
\hspace{0.5cm} Techniques for knowledge graph generation were heavily researched to identify the approach for this project. The tool ChatGPT \cite{chatgptwebsite} was explored for knowledge graph generation, but there was limited literature surrounding this topic due to its recentness. Nonetheless, \cite{chatgpt} was studied for introductory knowledge of the tool and investigation of its capabilities. Other tools such as SPARQL-Generate \cite{sparqlgenerate} were assessed using articles \cite{lefranccois2017flexible} and \cite{lefranccois2017sparql}, but found the overhead required to learn new syntax was too high. RML \cite{rml} was also considered using \cite{dimou2014rml} but was found to be more complex than existing solutions. All tools mentioned were adequate knowledge graph generation implements with guidance on how to proceed in articles and documentation. 

SPARQL Anything \cite{sparqlanythinggithub}, however, was the chosen direction for knowledge graph generation due to its simplicity and similarity to SPARQL 1.1 query language. Comparisons to other tools in the article \cite{sparqlanything} and \cite{asprino2023knowledge} provided stark rationale for selecting SPARQL Anything. This allowed use of SPARQL 1.1 query language resources such as a book \cite{sparlbook}, with an abundance of information regarding the language. Furthermore, drawbacks of RML and SPARQL-Generate included the lack of specific documentation surrounding the tools, which could pose a problem during debugging. Specific examples of knowledge graph generation with SPARQL Anything could also be seen in the documentation \cite{sparqlanythinggithub} and various other articles \cite{sparqlanything}, \cite{asprino2023knowledge} and \cite{rattaknowledge}. In terms of execution time and input size, all tools are relatively similar so any of the tools could have been used without major scalability challenges as discussed in \cite{sparqlanything}. 

Existing organ knowledge graphs were also researched. Resources surrounding this topic were sparse and the only relevant knowledge graphs were found on linked open dataset sites such as Wikidata \cite{organwikidata}, DBpedia \cite{organdbpedia} and MusicBrainz \cite{organmusicbrainz}. These resources provided insight into the current scope of available organ knowledge graphs, revealing basic material that needed to be fully explored. Using all available resources in our knowledge graph would be important in creating an expansive knowledge graph.

\section{Knowledge Graph Evaluation Techniques}
\hspace{0.5cm} Resources for critiquing and evaluating the produced knowledge graph included a book \cite{knowledgegraphevaulationbook} and a paper \cite{evaluationpaper}, which provided general information on how to evaluate the quality of any knowledge graph. Using two different mediums of evaluation information provided ample information on assessing knowledge graph quality. In particular, \cite{evaluationpaper}, provided reasoning and follow-up articles for quality assessors. Alternate in-depth methods of evaluation were explored in \cite{gao2019efficient}. However, aims of this article did not align with \cite{knowledgegraphevaulationbook} and \cite{evaluationpaper}, which mainly focused on the qualitative assessment of the produced knowledge graph. 

\section{Additional Complementary Literature}
\hspace{0.5cm} Any additional literature supported the report by offering more detail on said topics. For example, during implementation, JSON path was used so providing the consulted documentation \cite{jsonpath} would be helpful for those looking to extend the project or seek more detail. 
