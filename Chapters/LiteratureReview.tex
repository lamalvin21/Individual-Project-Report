\chapter{Literature Review}
In this chapter, resources (i.e. papers, websites, videos or books) reviewed are discussed. The literature studied covered a broad range of topics to help aid the understanding of the context and completion of the project.

\section{Background}
\hspace{0.5cm} Research surrounding the semantic web was required in order to provide a very broad context for the project and it's implementation. \cite{berners-TBLBook}, \cite{semanticweb}, \cite{rdf} and \cite{TTL} were all used. Various mediums of information were used in order to gain a comprehensive understanding of the context. Because WikiData \cite{wikidata} and MusicBrainz \cite{musicbrainz} could be used as potential external links in the implementation, getting a specific understanding of those topics was vital. 

Going into more detail about the project required contextual knowledge regarding ontologies and knowledge graphs. A foundation was built from the previous background research on the semantic web and RDF, so using articles such as \cite{ontology} and \cite{knowledgegraph} helped further and more relevant background information. Both articles introduced their topics well and for further research into the uses of knowledge graphs, using: \cite{searchengine}.   

The final section detailed information regarding SPARQL which was relevant to the tool being used for implementation as well as the general context. Resources used included a textbook\cite{sparlbook} and the FOAF Vocabulary Specification website \cite{foaf}, which was used to provide a simple example. 

\section{Context}
\hspace{0.5cm} Research regarding organs was necessary to gain an extensive understanding of the provided dataset. Starting with the history of organs required the use of \cite{organhistory}, \cite{organhistory1}, \cite{organmedivalhistory}. This involved a book as well as some online material to help provide a short historical summary of organs and their development over time. Extraction of relevant information from each source was necessary to provide a brief historical context. 

Understanding the different parts of an organ was important in the context of the provided dataset, so videos such as \cite{organvideo} and \cite{organvideo1}, gave a brief overview of the different components in an organ that may appear in the provided dataset. Using videos allowed visualisation of the different organ parts and it's mechanisms.   

The tool used to implement the solution was detailed using the GitHub repository documentation \cite{sparqlanythinggithub} as well as an article written by some of SPARQL Anything's contributors that provided reasoning for using this tool \cite{sparqlanything}. Both were used extensively during the implementation stage to guide the query construction as detailed information specifically surrounding SPARQL Anything was sparse. 

Details regarding an alternative solution: ChatGPT required background knowledge from it's official website \cite{chatgptwebsite} and an article that introduced ChatGPT and demonstrated it's various capabilities \cite{chatgpt}. Similar solutions: SPARQL-Generate \cite{sparqlgenerate} and RML \cite{rml} were explored through their documentation on their GitHub repositories. These sources in particular used their official websites as they were valid and reliable sources. 

The dataset was provided as part of a larger project: Polifonia, so details regarding it's purpose and contextual information was required from the project's specification \cite{polifoniaproject} as well as the official website \cite{polifonia}. These websites provided nuanced detail regarding the Polifonia project that allowed deeper understanding of the project's context. 

\section{Implementation}
\hspace{0.5cm} Using external links to expand the scope of the generated knowledge graph was essential. Specifically using the organ pages of both WikiData \cite{organwikidata} and MusicBrainz \cite{organmusicbrainz} was vital in the expansion of the generated knowledge graph. \cite{organdbpedia} could also be used but data on it was not as detailed as the selected choices and were often already covered.

\section{Evaluation}
\hspace{0.5cm} Resources for critiquing and evaluating the produced knowledge graph included a book \cite{knowledgegraphevaulationbook} and a paper \cite{evaluationpaper}, which provided general information on how to evaluate the quality of any knowledge graph. Using two different mediums of evaluation information provided ample information on assessing knowledge graph quality, in particular \cite{evaluationpaper}, which provided reasoning and follow-up articles for quality assessors. 

The built-in Measure-Command \cite{measurecommand} on Windows OS was also used to calculate time required to execute a command on command line and aided the evaluation of scalability. 

\section{Legal, Social, Ethical and Professional Issues}
\hspace{0.5cm} To assist with analysis of the Legal, Social, Ethical and Professional issues, the British Computer Society's Code of Conduct \cite{bcs} and an article regarding FAIR Principles \cite{fairprinciples} were referred to. Analysis of issues was carried out using the official documents to ensure no misinterpretation occurred. 

\section{Conclusion}
\hspace{0.5cm} Regarding extensions of the project, external datasets could be found in \cite{kaggle} and \cite{googledatasetsearch}, for example. Another possible extension similar to Wikidata and MusicBrainz was the organ site for DBpedia \cite{organdbpedia}. All were references particularly for the reader to carry out more research on these extensions if necessary. 