\chapter{Literature Review}
In this chapter, I will include any websites, papers, videos or textbooks I reviewed to help me in this project. The literature I studied covered a broad range of topics and grouped them into categories below. 

\section{Semantic Web}
\hspace{0.5cm} This section mainly involves websites, which I used for background research on different aspects of the semantic web.

\begin{itemize}
\item https://en.wikipedia.org/wiki/Semantic\_Web 
\item https://www.w3.org/standards/semanticweb/ 
\item https://en.wikipedia.org/wiki/Linked\_data
\item https://www.w3.org/RDF/
\item https://en.wikipedia.org/wiki/Resource\_Description\_Framework
\item http://xmlns.com/foaf/0.1/
\item https://en.wikipedia.org/wiki/FOAF
\item Resource Description Framework (RDF), Peter Loshin, February 2022
\item Wikidata: a new platform for collaborative data collection, April 2012, Denny Vrandečić
\end{itemize}


\section{Ontology and Knowledge Graphs}
\hspace{0.5cm} This section involves website links to help me understand what ontologies were and how they related to knowledge graphs. Researching them individually helped me gain context for the knowledge graph generation this project is about. 

\begin{itemize}
\item https://enterprise-knowledge.com/whats-the-difference-between-an-ontology-and-a-knowledge-graph/ 
\item https://www.ontotext.com/knowledgehub/fundamentals/what-are-ontologies/
\item https://en.wikipedia.org/wiki/Knowledge\_graph 
\end{itemize}


\section{Organs}
\hspace{0.5cm} This section provides context for the dataset, which is about organs. Understanding the different terms used in the dataset as well as the classes in the organ ontology was facilitated using the resources below. More websites had to be used in order to fully understand each class in the organ ontology, but the videos provided a great foundation. 

\begin{itemize}
\item https://en.wikipedia.org/wiki/Pipe\_organ 
\item https://youtu.be/4S6BErQs-HE 
\item https://youtu.be/kAlj3CE-7mM 
\item https://organhistoricalsociety.org/OrganHistory/works/works09.htm
\item https://www.britannica.com/art/organ-musical-instrument
\item https://organhistoricalsociety.org/OrganHistory/history/hist001.htm
\end{itemize}

\section{SPARQL-Anything}
\hspace{0.5cm} This section contains anything I used to help construct my query. Includes a SPARQL textbook as it familiarised me with the syntax used. 

\begin{itemize}
\item https://github.com/SPARQL-Anything/sparql.anything 
\item https://www.w3.org/TR/rdf-sparql-query/
\item Learning SPARQL 2ed: Querying and Updating with Sparql 1.1, August 2013, Bob DuCharme
\item https://arxiv.org/pdf/2106.02361.pdf
\end{itemize}

\section{Knowledge Graph Evaluation}
\hspace{0.5cm} This section contains the resources I used to discover how to critique and evaluate my completed knowledge graph. 
\begin{itemize}
\item https://www.emse.fr/~zimmermann/KGBook/Multifile/quality-assessment/
\item A Practical Framework for Evaluating the Quality of Knowledge Graphs - Haihua Chen, Gaohui Cao, Junhua Ding, Jiangping Chen (Jan 2019)
\end{itemize}
