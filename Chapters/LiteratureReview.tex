\chapter{Literature Review}
In this chapter, I will include resources (i.e. papers, websites, videos or books) I reviewed to help me in this project. The literature I studied covered a broad range of topics to help aid the understanding of the context and completion of the project.

\section{Background}
\hspace{0.5cm} Research surrounding the semantic web was required in order to provide a very broad context for the project and it's implementation. \cite{berners-TBLBook}, \cite{semanticweb}, \cite{rdf} and \cite{TTL} were all used. Various mediums of information were used in order to gain a comprehensive understanding of the context. Because WikiData \cite{wikidata} and MusicBrainz \cite{musicbrainz} could be used as potential external links in the implementation, getting a specific understanding of those topics was vital. 

Going into more detail about my project required contextual knowledge regarding ontologies and knowledge graphs. A foundation was built from the previous background research on the semantic web and RDF, so using articles such as \cite{ontology} and \cite{knowledgegraph} helped further and more relevant background information. Both articles introduced their topics well and for further research into the uses of knowledge graphs, I used: \cite{searchengine}.   

The final section detailed information regarding SPARQL which was relevant to the tool being used for implementation as well as the general context. Resources used included a textbook\cite{sparlbook} and the FOAF Vocabulary Specification website \cite{foaf}, which was used to provide a simple example. 

\section{Context}
\hspace{0.5cm} Research regarding organs was necessary to gain an extensive understanding of the provided dataset. Starting with the history of organs required the use of \cite{organhistory}, \cite{organhistory1}, \cite{organmedivalhistory}. Thhis involved a book as well as some online material to help provide a short historical summary of organs and their development over time. 

Understanding the different parts of an organ was important in the context of the provided dataset, so videos such as \cite{organvideo} and \cite{organvideo1}, gave me a brief overview of the different components in an organ that may appear in the provided dataset. 

The tool used to implement the solution was detailed using the GitHub repository documentation \cite{sparqlanythinggithub} as well as an article written by some of SPARQL-Anything's contributors that provided reasoning for using this tool \cite{sparqlanything}. Both were used extensively during the implementation stage to guide the query construction. 

Details regarding an alternative solution: ChatGPT required background knowledge from it's official website \cite{chatgptwebsite} and an article that introduced ChatGPT and demonstrated it's various capabilities \cite{chatgpt}. Similar solutions: SPARQL-Generate \cite{sparqlgenerate} and RML \cite{rml} were explored through their documentation on their github repositories.

The dataset was provided as part of a larger project: Polifonia, so details regarding it's purpose and contextual information was required from the project's specification \cite{polifoniaproject} as well as the official website \cite{polifonia}. 

\section{Implementation}
\hspace{0.5cm} Using external links to expand the scope of the generated knowledge graph was essential. Using the organ pages of both WikiData \cite{organwikidata} and MusicBrainz \cite{organmusicbrainz} was vital in the expansion of the generated knowledge graph. 

\section{Knowledge Graph Evaluation}
\hspace{0.5cm} This section contains the resources I used to discover how to critique and evaluate my completed knowledge graph. 
\begin{itemize}
\item https://www.emse.fr/~zimmermann/KGBook/Multifile/quality-assessment/
\item A Practical Framework for Evaluating the Quality of Knowledge Graphs - Haihua Chen, Gaohui Cao, Junhua Ding, Jiangping Chen (Jan 2019)
\end{itemize}
