\chapter{Implementation}
In this chapter, I will detail the implementation phase of the project and how I arrived at the end solution. Using the design section and following my specification, I will create my solution and mention any appropriate changes. 

% Testing based off requirements & specification
\section{Command Line Command}
\hspace*{0.5cm} Before starting implementation, familiarising myself with the command line commands was necessary to understand in order to commence implementation. Since the executable .jar file enables us to use SPARQL-Anything, starting the command with execution of the .jar file is needed. For example:

\lstset
{
    breaklines=true,
    breakatwhitespace=true,
    basicstyle=\ttfamily,
}
\begin{lstlisting}
    java -jar sparql-anything-0.8.0.jar 
\end{lstlisting}

\noindent Then, we can start expanding it to specify the file that contains the query by specifying it's location. For example:

\lstset
{
    breaklines=true,
    breakatwhitespace=true,
    basicstyle=\ttfamily,
}
\begin{lstlisting}
    -q filepath/filename.sparql
\end{lstlisting}

\noindent Because our project involves generating a knowledge graph of the user's choice, passing a parameter into the query is vital. It can be done as follows: 

\lstset
{
    breaklines=true,
    breakatwhitespace=true,
    basicstyle=\ttfamily,
}
\begin{lstlisting}
    --values variable=variablevalue
\end{lstlisting}

\noindent The assigned variable can be used in the SPARQL query by specifying the variable name by adding a question mark and underscore at the front. So passing the parameter 'variable', following the previous example, can be accessed using:

\lstset
{
    breaklines=true,
    breakatwhitespace=true,
    basicstyle=\ttfamily,
}
\begin{lstlisting}
    ?_variable
\end{lstlisting}

\noindent If necessary, the resulting knowledge graph can be output into a file given a folder location. For example:

\lstset
{
    breaklines=true,
    breakatwhitespace=true,
    basicstyle=\ttfamily,
}
\begin{lstlisting}
    --output filepath/outptufile.ttl
\end{lstlisting}

\noindent Altogether, the command will look like this:

\lstset
{
    breaklines=true,
    breakatwhitespace=true,
    basicstyle=\ttfamily,
}
\begin{lstlisting}
    java -jar sparql-anything-0.8.0.jar
    -q filepath/filename.sparql --values variable=variablevalue --output filepath/outptufile.ttl
\end{lstlisting}

\noindent In English, this command states: \\
\hspace*{0.5cm} "Using the executable file sparql-anything-0.8.0.jar, execute the query in filename.sparql from the filepath folder and pass in variable into the query with value variablevalue and output the knowledge in outputfile.ttl from the same filepath". 

\section{Prefixes}
\hspace*{0.5cm} The prefixes used in the ontology are mainly used in the relationship part of the triples. Below are the PREFIX that will be used:

\lstset
{
    breaklines=true,
    breakatwhitespace=true,
    basicstyle=\ttfamily,
}
\begin{lstlisting}
    PREFIX rdf:  <http://www.w3.org/1999/02/22-rdf-syntax-ns#>
    PREFIX rdfs: <http://www.w3.org/2000/01/rdf-schema#>
    PREFIX fx:   <http://sparql.xyz/facade-x/ns/>
    PREFIX xyz:  <http://sparql.xyz/facade-x/data/>
    PREFIX oont: <http://w3id.org/polifonia/ontology/organs/>
    PREFIX wd: <https://www.wikidata.org/wiki/> 
\end{lstlisting}

\section{Ontology Reconstruction}
\hspace*{0.5cm} As mentioned in the design, the scope of the provided ontology was too broad for the envisioned knowledge graph. Therefore, refinement of the provided ontology and exploration with external links was necessary to produce a relevant ontology. In the following subsections, I will detail the process of creating the ontology using the dataset and external data. This ontology will be put into the CONSTRUCT segment of the query which acts as the framework for the knowledge graph as mentioned in the context chapter. 

\subsection{Dataset Ontology}
\hspace*{0.5cm} Using the core structures outlined in my design, I identified and grouped relevant data to create an updated ontology. 

\subsubsection{Organ}
\hspace*{0.5cm} After selecting the most relevant information to the organ from each of the .json files that stores the data, I attached it to the organ node. Using the provided ontology to identify data relevant to the organ node was used as well to ensure the correct relationships were stated. Relevant data and relationship to the organ are listed below:

\begin{itemize}
    \itemsep0em 
    \item \textit{technicals} - xyz:technicals
    \item \textit{disposition} - xyz:disposition
    \item \textit{change} - xyz:change
    \item builder - oont:builder
    \item originallocation - oont:consolelocation
    \item dateOfBirth - oont:dateOfBirth
    \item building - oont:monument
    \item monumentNumber - oont:monumentNumber
    \item organName - oont:name
    \item organNumber - oont:organNumber 
    \item state - oont:state 
    \item particularity - oont:particularities
    \item history - oont:history
    \item creator - oont:creator
    \item moreInformation - oont:moreInformation 
\end{itemize}

This data is directly relevant to the organ as illustrated in the provided organ ontology. Data such as originallocation, dateOfbirth and state were added as they were part of the dataset and relevant to the organ. Because technicals, disposition and change were identified as nodes for further expansion, I originally connected them to the main organ node. 

The relationships used to describe the relation between the organ and it's objects were provided and usually had intuitive names usually based on the object. 

\subsubsection{Technicals}
\hspace*{0.5cm} This branch of technical data regarding the organ was mainly derived off one .json file and the section of the provided ontology that referred to it's 'parthood'. All relationships branching from this 'technicals' node were provided. Relevnat data and relationships identified are below:

\begin{itemize}
    \itemsep0em 
    \item systemPlayingAids - oont:systemPlayingAids
    \item pitch - oont:pitch
    \item range1 -  oont:keyboardRange
    \item range2 - oont:pedalRange
    \item temperature - oont:temperature
    \item windPressure - oont:windPressure
    \item windSystem - oont:windSystem
\end{itemize}

\textit{range1} and \textit{range2} corresponded to the ranges of the keyboard and pedal respectively so their relationship noted it as such. 

\subsubsection{Disposition}
\hspace*{0.5cm} This branch of data referred to the qualities of the selected organ. The provided ontology covered this section but had nodes that did not appear in the dataset so needed to be adjusted. Relationships were also provided. The readjustment of the ontology would include:

\begin{itemize}
    \itemsep0em 
    \item divisionname - xyz:divisionName
    \item partition - oont:partition
    \item specification - oont:AdditionalSpecification
\end{itemize}

\subsubsection{Change}
\hspace*{0.5cm} This branch stated the changes made to the organ during it's lifetime. The provided ontology did not mention changes of a given organ so I added a section in to represent it from the dataset into the knowledge graph. The relationships used were provided. The new section added to the ontology were:

\begin{itemize}
    \itemsep0em 
    \item datechange - oont:date 
    \item description - oont:AdditionalSpecification
    \item maintainer - oont:Builder
\end{itemize}

\subsection{External Ontology}
\hspace*{0.5cm} The method in which external data will be added to the knowledge graph will involve the ontology structure illustrated in the design section.

\subsubsection{External Data Sources}
\hspace*{0.5cm} As mentioned in the specification section, Wikidata and MusicBrainz can be used to expand the existing knowledge graph. I followed my design and created a node branching out of organ to represent Wikidata's organ page \cite{organwikidata} from which I mirrored the structure. The relationships used were the same as those on Wikidata. The data branching off the organwikidata node with their corresponding relationship were:

\begin{itemize}
    \itemsep0em 
    \item keyboardinstrument - wd:Property:P279
    \item buildingcomponent - wd:Property:P279
    \item organology - wd:Property:P2579 
    \item westernclassicalmusic - wd:Property:P366
    \item musictradition - wd:Property:P366
    \item organexpert - wd:Property:P3095
    \item organist - wd:Property:P1535
    \item catholicencyclopedia - wd:Property:P1343
    \item metropolitanmuseumofarttaggingvocabulary - wd:Property:P1343
    \item dbpedia - wd:Property:P1709
    \item organcase - wd:Property:P527
    \item organpipe - wd:Property:P527
    \item musicalkeyboard - wd:Property:P527
    \item pedalkeyboard - wd:Property:P527
    \item organstop - wd:Property:P527
    \item organconsole - wd:Property:P527
    \item swellbox - wd:Property:P527
    \item pipeorgan - wd:Property:P1889
\end{itemize}

The external data from MusicBrainz's organ page \cite{organmusicbrainz} was not as vast but extended the knowledge graph in a different way to Wikidata with overlapping data being connected. I expanded from the main organ node using relationships defined on MusicBrainz to provide more context with the relationships mirroring that of the MusicBrainz as well. This can be seen below:

\begin{itemize}
    \itemsep0em 
    \item barrelorgan - rdfs:subclassOf
    \item electricorgan - rdfs:subclassOf
    \item pipeorgan - rdfs:subclassOf
    \begin{itemize}
        \itemsep0em 
        \item pipeorganinfo - oont:extraInformation
    \end{itemize}
    \item reedorgan - rdfs:subclassOf
    \begin{itemize}
        \itemsep0em 
        \item reedorganimage - oont:locationImage
    \end{itemize}
    \item windinstrument - rdf:type
    \item organwikidata - 
\end{itemize}

\subsubsection{Custom Links}
\hspace*{0.5cm} These links used data from the dataset to form custom links, using string manipulation, as points of expansion. Identifying data from the dataset that was able to be expanded upon was difficult as data was very specific to the organ and external links online were too broad. However, data in the dataset that was more general such as locations could be expanded upon. Some points of expansion are:

\begin{itemize}
    \itemsep0em 
    \item buildinginfo
    \item stateinfo
    \item maintainerinfo
\end{itemize}

The relationship for all additional nodes is "oont:extraInformation".

\subsubsection{Generic Links}
\hspace*{0.5cm} These links are not organ-specific but provide more detail and general information regarding some aspects of the knowledge graph. The data in the dataset being so specific hinders the amount of external generic links that can be added, nevertheless, there are some points of expansion such as:

\begin{itemize}
    \itemsep0em 
    \item pitchinfo
    \item windPressureInfo
    \item divisionInfo
\end{itemize}

The relationship for all additional nodes is "oont:extraInformation".

\subsection{Final Ontology}
\hspace*{0.5cm} Combining both dataset and external data to form one ontology can be seen in RDF below:

\lstset
{
    breaklines=true,
    breakatwhitespace=true,
    basicstyle=\ttfamily,
}
\begin{lstlisting}
    ?organ a oont:Organ ;
        oont:builder ?builder ;
        oont:consolelocation ?originallocation ;
        oont:dateOfBirth ?dateOfBirth ;
        oont:monument ?building ;
        oont:monumentNumber ?monumentNumber ;
        oont:name ?organName ;
        oont:organNumber ?organNumber ;
        oont:state ?state ;
        oont:particularities ?particularity ;
        oont:history ?history ;
        oont:creator ?creator ;
        oont:moreInformation ?moreInformation ;
        xyz:technicals ?technicals ; 
        xyz:disposition ?disposition ;
        xyz:change ?change ;
        .

	?technicals oont:systemPlayingAids ?systemPlayingAids .
	?technicals oont:pitch ?pitch .
	?technicals oont:keyboardRange ?range1 . 
	?technicals oont:pedalRange ?range2 . 
	?technicals oont:temperature ?temperature .
	?technicals oont:windPressure ?windPressure .
	?technicals oont:windSystem ?windSystem .
	
	?disposition xyz:divisionName ?divisionname . 
	?disposition oont:partition ?partition .
	?disposition oont:AdditionalSpecification ?specification .

	?change oont:date ?datechange .
	?change oont:AdditionalSpecification ?changedescription . 
	?change oont:Builder ?maintainer .

	?building oont:extraInformation ?buildinginfo .
	?state oont:extraInformation ?stateinfo .
	?maintainer oont:extraInformation ?maintainerinfo .

	?pitch oont:extraInformation ?pitchinfo .
	?windPressure oont:extraInformation ?windPressureInfo .
	?windSystem oont:extraInformation ?windPressureInfo .
	?divisionname oont:extraInformation ?divisioninfo .

	?organ rdfs:subClassOf ?barrelorgan .
	?organ rdfs:subClassOf ?electricorgan .
	?organ rdfs:subClassOf ?pipeorgan .
	?organ rdfs:subClassOf ?reedorgan .
	?organ rdf:type ?windinstrument . 

	?pipeorgan oont:extraInformation ?pipeorganinfo .
	?reedorgan oont:locationImage ?reedorganimage .
	?organ oont:extraInformation ?organwikidata .

	?organwikidata wd:Property:P279 ?keyboardinstrument .
	?organwikidata wd:Property:P279 ?buildingcomponent .
	?organwikidata wd:Property:P2579 ?organology .
	?organwikidata wd:Property:P366 ?westernclassicalmusic .
	?organwikidata wd:Property:P366 ?musictradition .
	?organwikidata wd:Property:P3095 ?organexpert .
	?organwikidata wd:Property:P1535 ?organist .
	?organwikidata wd:Property:P1343 ?catholicencyclopedia .
	?organwikidata wd:Property:P1343 ?metropolitanmuseumofarttaggingvocabulary .
	?organwikidata wd:Property:P1709 ?dbpedia .

	?organwikidata wd:Property:P527 ?organcase .
	?organwikidata wd:Property:P527 ?organpipe .
	?organwikidata wd:Property:P527 ?musicalkeyboard .
	?organwikidata wd:Property:P527 ?pedalkeyboard .
	?organwikidata wd:Property:P527 ?organstop .
	?organwikidata wd:Property:P527 ?organconsole .
	?organwikidata wd:Property:P527 ?swellbox .

	?organwikidata wd:Property:P1889 ?pipeorgan .
\end{lstlisting}

This ontology can be placed in the CONSTRUCT section of the query.

\section{Knowledge Graph Generation}
\hspace*{0.5cm} The process of creating the knowledge graph generation query for my ontology will be detailed below:

\subsection{Dataset Knowledge Graph Query}
\hspace*{0.5cm} 


\subsection{External Knowledge Graph Query}
\hspace*{0.5cm} 


\subsection{Final Knowledge Graph Query}
\hspace*{0.5cm} 


\section{Resulting Knowledge Graph}
\hspace*{0.5cm} 




